    %
    \question[20] Primera Pregunta.
    \begin{solutionorbox}[3.5cm]
        Solución a la primera pregunta.
    \end{solutionorbox}
    
    %
    \question Segunda Pregunta.
    \begin{parts}
        \part[10] Pregunta 2 Sección a.
        \begin{solutionorbox}[3cm]
            Solución a la segunda pregunta-a.
        \end{solutionorbox}
        \part[10] Pregunta 2 Sección b.
        \begin{solutionorlines}[2.5cm]
            Solución a la segunda pregunta-b.
        \end{solutionorlines}
        \part[10] Pregunta 2 Sección c.
        \begin{solutionordottedlines}[2.5cm]
            Solución a la segunda pregunta-c.
        \end{solutionordottedlines}
    \end{parts}


    \question[10]
    Aproxime $\displaystyle \int_0^1 \sin x^2 \, dx$ hasta $0.001$ de su valor.
    \vspace{2mm}
    \begin{solutionorgrid}[3cm]
    \end{solutionorgrid}

    %
    \question[2] \fillin[Rojo][5cm] es mi color favorito.

    %
    \question[4] Preguntas alternativas múltiples en una línea.\\
    \begin{oneparcheckboxes}
        \choice John
        \choice Paul
        \choice George
        \choice Ringo
        \CorrectChoice Socrates
    \end{oneparcheckboxes}
    
    %
    \question[4] Preguntas con alternativas múltiples.
    \begin{choices}
        \choice John
        \choice Paul
        \choice George
        \choice Ringo
        \CorrectChoice Socrates
    \end{choices}
    
    
    \uplevel{Esta es la forma de agregar instrucciones entre preguntas y referenciarlas. Ej. ``Las preguntas \ref{exact-start} a la \ref{exact-end} deben se resueltas en detalle.''}
    \question[10]
    \label{exact-start}
    $\displaystyle \int_0^1 \frac{x^2 \, dx}{\sqrt{1-x^2}}$
    \question[10]
    $\displaystyle \int_0^1 \frac{1}{1+x^2}\, dx$
    \question[10]
    \label{exact-end}
    $\displaystyle \int_0^{\frac{\pi}{2}} \sin^3 x \cos x \, dx$

    \newpage
    \fillwithgrid{\stretch{1}}
