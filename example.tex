\documentclass[%
    11pt,
    letterpaper,
    addpoints,
%    answers        % Activar para mostrar las respuestas!
]{utfsm-exam}

\usepackage[utf8x]{inputenc}
%\usepackage[latin1]{inputenc}      % Windons (or MacOS with Mac Roman)
%\geometry{papersize={8.5in,13in},total={6.5in,11in}}   % Tamaño oficio

\newcommand{\curso}{Introducción al Latex}
\newcommand{\certamenNr}{\#1}      % ej. #1, #2, Recuparativo, Final
\newcommand{\programa}{Magister en Ciencias de la Ingeniería Industrial}
\newcommand{\fecha}{20 de junio de 2016}
\newcommand{\profesor}{Nombre del Profesor(a)}
\newcommand{\ayudante}{Nombre del Ayudante}

\begin{document}


\printHeader{}


%\centering
%    Este certamen tiene \numquestions\ preguntas, con un total de \numpoints\ puntos.


\begin{center}\footnotesize
    \gradetable[v][questions]
\end{center}


\instrucciones{Responda las preguntas en el espacio reservado para ello. No se corregirá fuera de estos márgenes.}


\begin{questions}
    \shadedsolutions
    %\bracketedpoints
    \pointpoints{Punto}{Puntos}
    \pointsinrightmargin
    \marginpointname{Pts.}
    
    
    %
    \question[20] Primera Pregunta.
    \begin{solutionorbox}[2cm]
        Solución a la primera pregunta.
    \end{solutionorbox}
    
    %
    \question Segunda Pregunta.
    \begin{parts}
        \part[10] Pregunta 2 Sección 1
        \begin{solutionorbox}[2.5cm]
            Solución a la primera pregunta.
        \end{solutionorbox}
        \part[10] Pregunta 2 Sección 2
        \begin{solutionorlines}[2.5cm]
            Solución a la primera pregunta.
        \end{solutionorlines}
        \part[10] Pregunta 2 Sección 3 
        \begin{solutionordottedlines}[2.5cm]
            Solución a la primera pregunta.
        \end{solutionordottedlines}
    \end{parts}


    \question[10]
    Aproxime $\displaystyle \int_0^1 \sin x^2 \, dx$ hasta $0.001$ de su valor.
    \vspace{2mm}
    \begin{solutionorgrid}[3cm]
    \end{solutionorgrid}

    %
    \question[2] \fillin[Rojo][5cm] es mi color favorito.

    %
    \question[4] Preguntas con alternativas múltiples.
    \begin{choices}
        \choice John
        \choice Paul
        \choice George
        \choice Ringo
        \CorrectChoice Socrates
    \end{choices}
    
    %
    \question[4] Preguntas alternativas múltiples en una línea.
    
    \begin{oneparcheckboxes}
        \choice John
        \choice Paul
        \choice George
        \choice Ringo
        \CorrectChoice Socrates
    \end{oneparcheckboxes}
        
    %

    
    \uplevel{Esta es la forma de agregar instrucciones entre preguntas y referenciarlas. Ej. ``Las preguntas \ref{exact-start} a la \ref{exact-end} deben se resueltas en detalle.''}
    \question[10]
    \label{exact-start}
    $\displaystyle \int_0^1 \frac{x^2 \, dx}{\sqrt{1-x^2}}$
    \question[10]
    $\displaystyle \int_0^1 \frac{1}{1+x^2}\, dx$
    \question[10]
    \label{exact-end}
    $\displaystyle \int_0^{\frac{\pi}{2}} \sin^3 x \cos x \, dx$
\end{questions}
\end{document}